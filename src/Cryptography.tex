\documentclass[a4paper, 12pt]{report}

\usepackage[dvipsnames]{xcolor}

%%%%%%%%%%%%%%%%
% Set Variables %
%%%%%%%%%%%%%%%%

\def\useItalian{0}  % 1 = Italian, 0 = English

\def\courseName{Cryptography}

\def\coursePrerequisites{TODO}

% \def\book{"My book",\\Author 1, ...}

% \def\authorName{Simone Bianco}
% \def\email{bianco.simone@outlook.it}
% \def\github{https://github.com/Exyss/university-notes}
% \def\linkedin{https://www.linkedin.com/in/simone-bianco}

\def\authorName{Alessio Bandiera}
\def\email{alessio.bandiera02@gmail.com}
\def\github{https://github.com/aflaag-notes}
\def\linkedin{https://www.linkedin.com/in/alessio-bandiera-a53767223}

% Do not change

%%%%%%%%%%%%
% Packages %
%%%%%%%%%%%%

\usepackage{../../packages/Nyx/nyx-packages}
\usepackage{../../packages/Nyx/nyx-styles}
\usepackage{../../packages/Nyx/nyx-frames}
\usepackage{../../packages/Nyx/nyx-macros}
\usepackage{../../packages/Nyx/nyx-title}
\usepackage{../../packages/Nyx/nyx-intro}

%%%%%%%%%%%%%%
% Title-page %
%%%%%%%%%%%%%%

\logo{../../packages/Nyx/logo.png}

\if\useItalian1
    \institute{\curlyquotes{\hspace{0.25mm}Sapienza} Università di Roma}
    \faculty{Ingegneria dell'Informazione,\\Informatica e Statistica}
    \department{Dipartimento di Informatica}
    \ifdefined\book
        \subtitle{Appunti integrati con il libro \book}
    \fi
    \author{\textit{Autore}\\\authorName}
\else
    \institute{\curlyquotes{\hspace{0.25mm}Sapienza} University of Rome}
    \faculty{Faculty of Information Engineering,\\Informatics and Statistics}
    \department{Department of Computer Science}
    \ifdefined\book
        \subtitle{Lecture notes integrated with the book \book}
    \fi
    \author{\textit{Author}\\\authorName}
\fi

\title{\courseName}
\date{\today}

% \supervisor{Linus \textsc{Torvalds}}
% \context{Well, I was bored\ldots}

\addbibresource{./references.bib}

%%%%%%%%%%%%
% Document %
%%%%%%%%%%%%

\begin{document}
\maketitle

% The following style changes are valid only inside this scope 
{
	\hypersetup{allcolors=black}
	\fancypagestyle{plain}{%
		\fancyhead{}        % clear all header fields
		\fancyfoot{}        % clear all header fields
		\fancyfoot[C]{\thepage}
		\renewcommand{\headrulewidth}{0pt}
		\renewcommand{\footrulewidth}{0pt}}

	\romantableofcontents
}

\introduction

%%%%%%%%%%%%%%%%%%%%%

\chapter{TODO}

\section{TODO}

TODO \todo{missing introduction}

\begin{frameddefn}{Perfect secrecy}
	Given any distribution $M$ over $\mathcal M$, and $k$ chosen UAR on $\mathcal K$, we say that $\Pi = (\mathrm{Enc}, \mathrm{Dec})$ is \tbf{perfectly secret} if $$\forall m \in \mathcal M, c \in \mathcal C \quad \Pr[M = m] = \Pr[M = m \mid C = c]$$
\end{frameddefn}

TODO \todo{de sta definizione non me torna la distribuzione} In other words, this definition requires the encrypted text $c$ to \tit{not reveal} anything about the plaintext $m$. The following lemma shows some properties about perfect secrecy.

\begin{framedlem}{}
	The following three conditions are equivalent:

	\begin{enumerate}
		\item perfect secrecy
		\item independence of $M$ and $C$
		\item $\forall m, m' \in \mathcal M, c \in C \quad \Pr_{k \in \mathcal K}[\mathrm{enc}(k, m) = c] = \Pr_{k \in \mathcal K}[\mathrm{enc}(k, m') = c]$
	\end{enumerate}
\end{framedlem}

\begin{proof}
	We will prove the statements cyclically.

	\begin{itemize}
		\item $1 \implies 2$. By perfect secrecy, we have that $$\Pr[M = m] = \Pr[M = m \mid C = c] = \dfrac{\Pr[M = m \land C = c]}{\Pr[C = c]}$$ therefore, by rearranging the terms we get that $$\Pr[M = m \land C = c] = \Pr[M = m] \cdot \Pr[C = c]$$
		\item $2 \implies 3$. Fix $m \in \mathcal M$ and $c \in \mathcal C$; we have that

		      \begin{equation*}
			      \begin{alignedat}{2}
				      \Pr_{k \in \mathcal K}[\mathrm{enc}(k, m) = c] & = \Pr_{k \in \mathcal K}[\mathrm{enc}(K, M) \mid M = m] &                                                     \\
				                                                     & = \Pr_{k \in \mathcal K}[C = c \mid M = m]              & \quad \quad (\mbox{by definition})                  \\
				                                                     & = \Pr[C = c]                                            & \quad \quad (\mbox{by independence of $M$ and $C$})
			      \end{alignedat}
		      \end{equation*}

		      Now fix another message $m' \in \mathcal M$; we can repeat the same steps and obtain that $\Pr_{k \in \mathcal K}[\mathrm{enc}(k, m') = c] = \Pr[C = c]$ which concludes the proof.
		\item $3 \implies 1$. Fix $c \in \mathcal C$.

		      \claim{
			      $\Pr[C = c] = \Pr[C = c \mid M = m]$
		      }{
			      By assuming property 3, we get that

			      \begin{equation*}
				      \begin{alignedat}{2}
					      \Pr[C = c] & = \sum_{m' \in \mathcal M}{\Pr[C = c \mid M = m'] \cdot \Pr[M = m']}                                     & \quad (\mbox{by the L.T.P.}) \\
					                 & = \sum_{m' \in \mathcal M}{\Pr_{k \in \mathcal K}[\mathrm{enc}(k, M) = c \mid M = m'] \cdot \Pr[M = m']} &                              \\
					                 & = \sum_{m' \in \mathcal M}{\Pr_{k \in \mathcal K}[\mathrm{end}(k, m') = c] \cdot \Pr[M = m']}            &                              \\
					                 & = \sum_{m' \in \mathcal M}{\Pr_{k \in \mathcal K}[\mathrm{end}(k, m) = c] \cdot \Pr[M = m']}             & \quad (\mbox{by property 3}) \\
					                 & = \Pr_{k \in \mathcal K}[\mathrm{enc}(k, m) = c] \cdot \sum_{m' \in \mathcal M}{\Pr[m = m']}             &                              \\
					                 & = \Pr_{k \in \mathcal K}[\mathrm{enc}(k, m) = c]                                                         &                              \\
					                 & = \Pr_{k \in \mathcal K}[\mathrm{enc}(k, M) = c \mid M = m]                                              &                              \\
					                 & = \Pr_{k \in \mathcal K}[C = c \mid M = m]                                                               &                              \\
				      \end{alignedat}
			      \end{equation*}
		      }

		      Finally, by Bayes' theorem we have that

		      \begin{equation*}
			      \begin{alignedat}{2}
				      \Pr[M = m] & = \dfrac{\Pr[M = m \mid C = c] \cdot \Pr[C = c]}{\Pr[C = c \mid M = m]} &                             \\
				                 & = \Pr[M = m \mid C = c]                                                 & \quad (\mbox{by the claim}) \\
			      \end{alignedat}
		      \end{equation*}

		      which is precisely perfect secrecy.
	\end{itemize}
\end{proof}

TODO \todo{buco OTP}

% \printbibliography % UNCOMMENT FOR BIBLIOGRAPHY

\end{document}
